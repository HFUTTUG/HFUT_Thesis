% !TeX encoding = UTF-8
% !TeX program = xelatex
% !TeX spellcheck = en_US

%
% Copyright (C) 2015-2021 by USTC TeX Users Group <https://github.com/ustctug>
%
% hfutthesis developed by HFUT TeX Users Group <https://github.com/HFUTTUG> on the 
% base of ustcthesis <https://github.com/usthtug/ustcthesis>.

% This file may be distributed and/or modified under the
% conditions of the LaTeX Project Public License, either version 1.3c
% of this license or (at your option) any later version.
% The latest version of this license is in
%    https://www.latex-project.org/lppl.txt
% and version 1.3c or later is part of all distributions of LaTeX
% version 2008 or later.
%.

\documentclass[a4paper]{ltxdoc}

\usepackage[UTF8]{ctex}
\usepackage{unicode-math}
\usepackage{caption}
\usepackage{booktabs}
\usepackage{xcolor}
\usepackage{listings}
\usepackage[perpage]{footmisc}
\usepackage{hypdoc}
\usepackage{bicaption}
\makeatletter

% 设置字体
\IfFileExists{/System/Library/Fonts/Times.ttc}{
  \setmainfont{Times}
  \setsansfont[Scale=MatchLowercase]{Helvetica}
  \setmonofont[Scale=MatchLowercase]{Menlo}
  \xeCJKsetwidth{‘’“”}{1em}
}{}
\unimathsetup{
  math-style=ISO,
  bold-style=ISO,
}
\IfFontExistsTF{xits-math.otf}{
  \setmathfont[
    Extension    = .otf,
    BoldFont     = *bold,
    StylisticSet = 8,
  ]{xits-math}
  \setmathfont[range={cal,bfcal},StylisticSet=1]{xits-math.otf}
}{
  \setmathfont[
    Extension    = .otf,
    BoldFont     = XITSMath-Bold,
    StylisticSet = 8,
  ]{XITSMath-Regular}
  \setmathfont[range={cal,bfcal},StylisticSet=1]{XITSMath-Regular.otf}
}

% 定义一些命令用于写文档
\newcommand\TeXLive{\TeX{} Live}
\newcommand\unicodechar[1]{U+#1(\symbol{"#1})}
\DeclareRobustCommand\file{\nolinkurl}
\DeclareRobustCommand\env{\texttt}
\DeclareRobustCommand\pkg{\textsf}
\DeclareRobustCommand\cls{\textsf}
\DeclareRobustCommand\opt{\texttt}

% 在 doc 的基础上增加 option 的描述
\def\DescribeOption{\leavevmode\@bsphack\begingroup\MakePrivateLetters
  \Describe@Option}
\def\Describe@Option#1{\endgroup
              \marginpar{\raggedleft\PrintDescribeOption{#1}}%
              \SpecialEnvIndex{#1}\@esphack\ignorespaces}
\@ifundefined{PrintDescribeOption}
   {\def\PrintDescribeOption#1{\strut \MacroFont #1\ }}{}

% 调整列表的格式
\setlength\partopsep{\z@}
\def\@listi{\leftmargin\leftmargini
            \parsep \z@
            \topsep \z@
            \itemsep \z@}
\let\@listI\@listi
\@listi

% listings 的样式
\lstdefinestyle{lstshell}{
  basicstyle      = \small\ttfamily,
  backgroundcolor = \color{lightgray},
  language        = bash,
}
\newcommand\shellcmd[1]{\colorbox{lightgray}{\lstinline[style=lstshell]|#1|}}
\lstnewenvironment{shell}{\lstset{style=lstshell, gobble=2}}{}
\lstnewenvironment{latex}{%
  \lstset{
    basicstyle = \small\ttfamily,
    frame      = single,
    gobble     = 2,
    language   = [LaTeX]TeX,
  }%
}{}

\hypersetup{
  allcolors         = blue,
  bookmarksnumbered = true,
  bookmarksopen     = true,
}
\makeatother


\begin{document}



\title{\cls{hfutthesis} 使用说明}
\author{
  \href{https://github.com/HFUTTUG}{HFUTTUG}
 }
\date{v1.0.4\qquad 2022-04-09}
\maketitle



\section{简介}

本模板 \cls{hfutthesis} 是合肥工业大学本科生和研究生学位论文的 \LaTeX{}
模板, 按照最新版的《\href{http://xwgl.hfut.edu.cn/2021/0419/c1975a253949/page.htm}
{合肥工业大学研究生学位论文写作规范}》和
《\href{http://xcjwb.hfut.edu.cn/53/cc/c1144a21452/page.htm}
{合肥工业大学本科毕业设计(论文)工作实施细则}》(以下统称《规范》)的要求编写,兼容最新版的 TeX Live、MacTeX 、MiKTeX 发行版,支持跨平台使用。

本项目在 \href{https://github.com/ustctug/ustcthesis}
{ustctug/ustcthesis} 基础上编写,\href{https://github.com/LaureatePoet}{LaureatePoet}和\href{https://github.com/HowardZorn}{Fw[a]rd} 基于合肥工业大学本科生和研究生学位论文的相关要求,开发了本模板的 v1.0.0 版本。

下载地址:
\begin{itemize}
  \item GitHub Releases:\url{https://github.com/HFUTTUG/HFUT_Thesis/releases}
  \item Overleaf 模板:\url{https://www.overleaf.com/latex/templates/hfut-thesis/ythwdqdkpvkp}
\end{itemize}

用户在使用 \pkg{hfutthesis} 模板前,应先阅读学校关于毕业论文的相关规定。
如果在使用的过程中遇到问题,可以阅读
\href{https://github.com/HFUTTUG/HFUT_Thesis/wiki}{常见问题},
或者在 \href{https://github.com/HFUTTUG/HFUT_Thesis/issues}{GitHub Issues}
中反馈。

\section{编译方法}


\subsection{文件组成}
本模板的主要文件如表~\ref{tab:files}:
\begin{table}[htb]
  \centering\small
  \caption{模板的文件组成}
  \label{tab:files}
  \begin{tabular}{lll}
    \toprule
    类别     & 文件                      & 说明                         \\
    \midrule
    模板文件 & \file{hfutthesis.cls}     & 模板文件                     \\
             & \file{hfutthesis-*.bst}   & \BibTeX{} 参考文献表格式     \\
             & \file{hfutthesis-*.bbx}   & BibLaTeX 参考文献表格式      \\
             & \file{hfutthesis-*.cbx}   & BibLaTeX 参考文献表格式      \\
             & \file{figures/*.pdf} & 校名和校徽图片               \\
    \midrule
    使用说明 & \file{hfutthesis-doc.tex} & 模板使用说明的源代码         \\
             & \file{hfutthesis-doc.pdf} & (你正在阅读的)模板使用说明 \\
    \midrule
    示例文档 & \file{main.tex}           & 主文档                       \\
             & \file{hfutsetup.tex}      & 配置文件                     \\
             & \file{chapters/*.tex}     & 示例文档的各个章节           \\
             & \file{figures/}           & 放置图片的目录               \\
             & \file{bib/hfut.bib}       & \BibTeX{} 示例数据库         \\
    \midrule
    其他     & \file{README.md}          & 基本说明                     \\
             & \file{HFUT_Thesis.json}       & 更新日志                 \\
             & \file{latexmkrc}          & latexmk 的配置文件           \\
             & \file{Makefile}           & GNU make 的配置文件          \\
             & \file{.vscode/}           & VS Code 的配置文件           \\
             & \file{build.lua}          & l3build 的配置文件           \\
    \bottomrule
  \end{tabular}
\end{table}

示例文档包括了常用的 \LaTeX{} 命令,建议新手从此入手,用自己的内容进行替换。


\subsection{依赖宏包}

本模板要求使用 TeX Live、MacTeX 或 MiKTeX 不低于 2017 年的发行版,
推荐升级到最新的版本。

模板直接依赖的宏包有:
\pkg{amsthm},
\pkg{graphicx},
\pkg{booktabs},
\pkg{longtable},
\pkg{algorithm2e},
\pkg{siunitx},
\pkg{natbib},
\pkg{hyperref}。

另外,模板还对其他宏包提供了支持,包括:
\pkg{amsmath},
\pkg{biblatex},
\pkg{nomencl}等。
这些宏包并非必需,用户可以根据需要选择使用。
模板在检测到这些宏包被调用后会自动进行配置。

注意,本模板\emph{不}兼容的宏包有:
\pkg{amsfonts},
\pkg{amssymb},
\pkg{bm},
\pkg{cite},
\pkg{mathrsfs},
\pkg{newtx},
\pkg{upgreek}。


\subsection{开始编译}

\begin{enumerate}

\item GNU make \\
Linux/Mac用户,可以直接使用 GNU make 工具,这是最简单的方法。
编译论文 \file{main.pdf}:
\begin{shell}
  make
\end{shell}
编译说明文档 \file{hfutthesis-doc.pdf}:
\begin{shell}
  make doc
\end{shell}
另外还可以用 \shellcmd{make clean} 清理辅助文件。

\item |latexmk| \\
Windows 用户可能无法使用GNU make,使用 |latexmk| 也是一个比较简单的方法,
配置文件由 \file{latexmkrc} 给出,其参数设置为 |-xelatex|,用户编译论文
只需使用命令:
\begin{shell}
  latexmk -xelatex main.tex
\end{shell}
编译说明文档:
\begin{shell}
  latexmk -xelatex hfuthesis-doc.tex
\end{shell}
清理辅助文件可以用 \shellcmd{latexmk -c}。图形界面用户应参考编辑器的使用说明。

\item 手动编译 \\
手动编译是最繁琐的方法,用户可能需要运行多遍,以确保论文的交叉引用等信息全部正确。

编译论文 \file{main.pdf}:
\begin{shell}
  xelatex main.tex
  bibtex main.aux
  xelatex main.tex
  xelatex main.tex
\end{shell}
编译说明文档 \file{hfutthesis-doc.pdf}:
\begin{shell}
  xelatex hfutthesis-doc.tex
  xelatex hfutthesis-doc.tex
\end{shell}
\end{enumerate}



\section{模板设置}


\subsection{文档类参数}
\DescribeOption{degree}
选择学位,支持 \opt{bachelor}(默认),\opt{master},\opt{doctor}。
\begin{latex}
  \documentclass[degree=bachelor]{hfutthesis}
\end{latex}

\DescribeOption{degree-type}
学位类型。可选:学术型 \opt{academic}(默认),专业型 \opt{professional}。
\begin{latex}
  \documentclass[degree-type=professional]{hfutthesis}
\end{latex}

\DescribeOption{language}
论文全文的主要语言。可选:\opt{chinese}(默认),\opt{english}。
\begin{latex}
  \documentclass[language=english]{hfutthesis}
\end{latex}

\DescribeOption{output}
输出 PDF 的类型:
\begin{itemize}
  \item \opt{print}(默认):用于双面打印纸质论文
  \item \opt{electronic}:单面,并保留超链接颜色
\end{itemize}
\begin{latex}
  \documentclass[output=electronic]{hfutthesis}
\end{latex}

\DescribeOption{section-style}
章节标题的样式。可选:\opt{chinese}(默认),\opt{english}。
\begin{itemize}
  \item \opt{chinese}:汉字序号
  \item \opt{arabic}:数字序号
\end{itemize}
\begin{latex}
  \documentclass[section-style=arabic]{hfutthesis}
\end{latex}


\subsection{字体设置}
模板默认会自动根据操作系统配置合适的字体,
\DescribeOption{fontset}
用户也可以通过 \opt{fontset} 时指定使用预设的字库,如:
\begin{latex}
  \documentclass[fontset=windows]{hfutthesis}
\end{latex}
允许的选项有 \opt{auto}(默认)、\opt{windows}、\opt{mac}、\opt{ubuntu}、\opt{fandol}和\opt{none}。
具体使用的字体见表~\ref{tab:fontset}。
用户也可以设置为 \opt{none} 并自行配置字体。

\begin{table}[htb]
  \centering
  \caption{模板预设的字体}
  \label{tab:fontset}
  \begin{tabular}{cccc}
    \toprule
    \opt{windows} & \opt{mac}    & \opt{ubuntu} & \opt{fandol} \\
    \midrule
    Times New Roman  & Times New Roman & TeX Gyre Termes & TeX Gyre Termes \\
    SimSun            & SimSun           & TeX Gyre Heros  & TeX Gyre Heros  \\
    Courier          & Menlo           & TeX Gyre Cursor & TeX Gyre Cursor \\
    中易宋体         & 华文宋体        & 思源宋体        & Fandol 宋体     \\
    中易黑体         & 华文黑体        & 思源黑体        & Fandol 黑体     \\
    \bottomrule
  \end{tabular}
\end{table}

然而自动配置的字体只能保证编译通过,但是还存在一些问题:
\begin{enumerate}
  \item 在其他平台上配置的 TeX Gyre 系列字体,虽然在风格上比较接近 Times 和 Arial,
    但是毕竟跟《规范》的字体不完全一致;
  \item Fandol 字库的字形较少,常常出现缺字的情况;
  \item 华文字库和 Fandol 字库虽然不违反《规范》的要求,
    但是其字形跟中易字库有所差别,可能被审查老师认为格式不符合要求。
\end{enumerate}

所以建议在提交最终版前使用 Windows 平台的字体进行编译。

\subsection{更多设置}
\DescribeOption{\textbackslash hfut@define@key}
本模板对接口进行了初始化定义,如有更个性的需求,可在\file{hfutthesis.cls}中查看接口及缺省值并进行修改,如:
\begin{latex}
  \hfut@define@key{
  degree = {
    choices = {
      doctor,
      master,
      bachelor,
    },
    default = bachelor,
  },
  ...
  }
\end{latex}
修改\opt{default}的值为\opt{doctor}、\opt{master}和\opt{bachelor},即可修改学位默认类型。

\begin{latex}
  \hfut@define@key{
  ugtype = {
    default = {论文},
  },
  ... 
  }
\end{latex}
修改\opt{default}的值为“其他名称”,ugtype的默认值即修改为“其他名称”。


\section{论文内容}
\subsection{内容顺序}
根据《规范》,研究生论文和本科生论文在内容的要求上有所差异,用户需要根据\file{main.tex}中的注释提示进行调整。

研究生论文的内容按照以下顺序排列:

\begin{description}
  \item[maketitle] 封面,原创性声明及授权使用说明
  \item[frontmatter] 致谢,摘要,目录,图、表清单,(符号说明)
  \item[mainmatter] 正文章节,参考文献
  \item[appendix] 附录
  \item[backmatter] 已发表论文列表
\end{description}

本科生论文的内容按照以下顺序排列:

\begin{description}
  \item[maketitle] 封面,原创性声明和授权使用声明
  \item[frontmatter] 摘要,目录,图、表清单,(符号说明)
  \item[mainmatter] 正文章节,参考文献
  \item[backmatter] (已发表论文列表),致谢
  \item[appendix] 附录
\end{description}

注:
\begin{itemize}
  \item  示例文档 \file{main.tex} 中的致谢、目录等章节的顺序,是按照本科生论文的格式组织内容的,因此研究生需要手动调整顺序。
  \item  在某些部分中,本科生、硕士生和博士生的内容要求有细微的差异,需根据\file{main.tex}中的注释进行修改。例如,博士学位论文在原创性声明前有专家清单页,需去除注释。
\end{itemize}
 


\subsection{封面}

“封面”的名字让人有些混淆,它既可以指由印刷厂统一制作的硬皮封面(cover),也可
以指书打开后的第一页(title page)。在这里指的是后者,所以本模板从 title page
开始。

\DescribeMacro{\maketitle}
\DescribeMacro{\copyrightpage}
封面和声明页分别由 \cs{maketitle} 和 \cs{copyrightpage} 命令生成,
其中的各项信息在\file{hfutsetup.tex}中的\cs{hfutsetup} 中修改,如:
\begin{latex}
  \hfutsetup{
    title  = {论文中文题目},
    title* = {Thesis English Title},
  }
\end{latex}
模板提供的选项见表~\ref{tab:covercmds}(部分),
\begin{table}[htb]
  \centering\small
  \caption{录入封面信息的命令}
  \label{tab:covercmds}
  \begin{tabular}{lll}
    \toprule
    命令                & 命令(英文)         & 说明       \\
    \midrule
    \opt{title}         & \opt{title*}         & 论文标题   \\
    \opt{stuID}         & -                    & 学号   \\
    \opt{author}        & \opt{author*}        & 作者姓名   \\
    \opt{speciality}    & \opt{speciality*}    & 学科专业   \\
    \opt{supervisor}    & \opt{supervisor*}    & 导师姓名   \\
    \opt{secret-level}  & \opt{secret-level*}  & 密级       \\
    \opt{secret-year}   & -                    & 保密年限   \\
    \bottomrule
  \end{tabular}
\end{table}

有几点说明:
\begin{itemize}
  \item \cs{hfutsetup} 使用 \pkg{kvsetkeys} 机制,配置项之间不能有空行,否则会报错。
  \item 其中带 |*| 后缀的选项用于设置英文封面。
  \item 导师姓名 \opt{supervisor} 允许多个姓名,使用“,”(西文逗号 U+002C)隔开。
  \item 完成时间 \opt{date}默认为当前日期,如需更改请取消注释并使用 ISO 格式。
\end{itemize}


\subsection{摘要和章节}
对于特殊的章节,\cls{hfutthesis} 还提供了相应的环境:
\begin{itemize}
  \item 中文摘要:\env{abstract}
  \item 英文摘要:\env{abstract*}
  \item 符号说明:\env{notation}
  \item 致谢:    \env{acknowledgements}
  \item 发表成果:\env{publications}
\end{itemize}

摘要的关键词应使用 \cs{hfutsetup} 的接口进行设置,
只要在摘要结束前即可,比如:
\begin{latex}
  \begin{abstract}
    这里是摘要。
    \hfutsetup{keywords = {论文;摘要;关键词}}
  \end{abstract}
\end{latex}

目录和图、表清单可以使用命令自动生成:
\begin{itemize}
  \item 目录:  \cs{tableofcontents}
  \item 图清单:\cs{listoffigures}
  \item 表清单:\cs{listoftables}
\end{itemize}


\subsection{浮动体}

学士学位论文要求图表中文标题,请使用\cs{caption}\marg{中文标题};硕博学位论文要求图表中英文标题,请使用\cs{bicaption}\marg{\{中文标题\}\{English title\}}。例如

\begin{table}[h]
  \centering
   \caption{中文标题} % 默认:学士学位论文 图表中文标题
  \label{tab:exampletable-zh}
  \begin{tabular}{cl}
    \toprule
    类型   & 描述                                       \\
    \midrule
    挂线表 & 挂线表也称系统表、组织表,用于表现系统结构 \\
    无线表 & 无线表一般用于设备配置单、技术参数列表等   \\
    卡线表 & 卡线表有完全表,不完全表和三线表三种       \\
    \bottomrule
  \end{tabular}
\end{table}
\begin{table}[h]
  \centering
  \bicaption{中文标题}{English Title} % 硕博学位论文 图表中英文标题
   \label{tab:exampletable-zh-en}
  \begin{tabular}{cl}
    \toprule
    类型   & 描述                                       \\
    \midrule
    挂线表 & 挂线表也称系统表、组织表,用于表现系统结构 \\
    无线表 & 无线表一般用于设备配置单、技术参数列表等   \\
    卡线表 & 卡线表有完全表,不完全表和三线表三种       \\
    \bottomrule
  \end{tabular}
\end{table}
\DescribeMacro{\note}
本模板还提供了 \cs{note}\marg{notes} 命令,用于在图表中添加注释。

关于图片的并排,推荐使用较新的 \pkg{subcaption} 宏包,不建议使用
\pkg{subfigure} 或 \pkg{subfig}。

更多的表格样式参见 \pkg{booktabs}(三线表)、\pkg{longtable}(跨页表格)。

算法可以使用 \pkg{algorithms} 宏包或者较新的 \pkg{algorithm2e}。

\subsection{数学}

《规范》要求数字和字母统一采用 Times New Roman体,与\TeX{} 默认的美国数学学会(AMS)的使用习惯可能有所出入。

\DescribeOption{math-font}
模板使用默认使用 XITS Math 作为数学字体。
用户也可以使用 \opt{math-font} 选项切换其他数学字体,可选:
\opt{stix}(STIX Two Math)、
\opt{libertinus}(Libertinus Math)、
\opt{newcm}(New Computer Modern Math)、
\opt{cm}(Computer Modern Math)、
\opt{lm}(Latin Modern Math)。

其中 \opt{lm} 和 \opt{newcm} 的字形比较搭配 TeX 原生的 Computer Modern 字体,
但与《规范》要求的西文字体 Times New Roman 并不搭配。
可能会造成正文和公式中的数字字体不一致,需要谨慎使用。

\DescribeOption{math-style}
用户可以通过设置 \opt{math-style} 选择数学符号样式(可选:
\opt{GB}(中文默认),\opt{TeX}(英文默认)和 \opt{ISO}),比如:
\begin{latex}
  \hfutsetup{
    math-style = ISO,
  }
\end{latex}

用户也可以逐项修改数学样式。
\newcommand\dif{\mathop{}\!\mathrm{d}}
\begin{enumerate}
  \item \DescribeOption{uppercase-greek}
    大写希腊字母的正/斜体,可选:\opt{italic}、\opt{upright}。
    有限增量符号 $\increment x$ 固定使用正体,推荐使用 \cs{increment} 表示。
  \item \DescribeOption{less-than-or-equal}
    小于等于号和大于等于号的字形,可选:\opt{slanted}、\opt{horizontal}。
    这将控制 \cs{le}、\cs{ge}、\cs{leq} 和 \cs{geq} 的符号
    是“$\leqslant$、$\geqslant$”还是“$\leq$、$\geq$”。
  \item \DescribeOption{integral}
    积分号的正/斜体,可选:\opt{upright}、\opt{slanted}。
    该选项需要字体的支持,目前仅限 \opt{xits}、\opt{stix}、
    \opt{libertinus} 和 \opt{newcm}。参考下文关于数学字体的选择。
  \item \DescribeOption{integral-limits}
    积分号上下限的位置,可选:\opt{true}(在上下)、\opt{false}(在右侧)。
    这个设置只影响行间公式,行内公式统一居右侧,不受影响。
  \item \DescribeOption{partial}
    偏微分符号的正/斜体,可选:\opt{upright}、\opt{slanted}。
  \item \DescribeOption{math-ellipsis}
    省略号 \cs{dots} 的样式,可选:\opt{centered}(按照中文的习惯固定居中)、
    \opt{lower} 和 \opt{AMS}(取决于前后符号的位置)。
    其他的省略号命令如 \cs{lots}、\cs{cdots} 则不受影响。
  \item \DescribeOption{real-part}
    实部 \cs{Re} 和虚部 \cs{Im} 的字体,可选:\opt{roman} 和 \opt{fraktur}。
\end{enumerate}

如果数学符号选择国标样式 |math-style = GB|,相当于设置了
\begin{latex}
  \hfutsetup{
    uppercase-greek    = italic,
    less-than-or-equal = slanted,
    integral           = upright,
    integral-limits    = true,
    partial            = upright,
    math-ellipsis      = centered,
    real-part          = roman,
  }
\end{latex}

另外,国标的数学样式与 AMS 还有些差异,但无法在模板中统一设置。用户需要在写作时进行处理。
\begin{enumerate}
  \item 数学常数和特殊函数名用正体,如 $\uppi = 3.14\dots$;$\symup{i}^2 = -1$;
    $\symup{e} = \lim_{n \to \infty} \left( 1 + \frac{1}{n} \right)^n$。
  \item 微分号使用正体,比如 $\dif y / \dif x$。
  \item 向量、矩阵和张量用粗斜体(\cs{symbf}),如 $\symbf{x}$、$\symbf{\Sigma}$、$\symbfsf{T}$。
  \item 自然对数用 $\ln x$ 不用 $\log x$。
\end{enumerate}

需要注意,上述关于数学符号风格的设置在设置数学字体(\opt{math-font})时才会生效。

以上字体都是 OpenType 格式的字体,需要配合
\href{http://mirrors.ctan.org/macros/latex/contrib/unicode-math/unicode-math.pdf}{\pkg{unicode-math}}
宏包使用。
全部数学符号的命令参考
\href{http://mirrors.ctan.org/macros/latex/contrib/unicode-math/unimath-symbols.pdf}{\pkg{unimath-symbols}}。
注意,\pkg{unicode-math} 宏包与 \pkg{amsfonts}、\pkg{amssymb}、\pkg{bm}、
\pkg{mathrsfs}、\pkg{upgreek} 等宏包\emph{不}兼容。
模板作了处理,用户可以直接使用这些宏包的命令,如 \cs{bm}、\cs{mathscr}、
\cs{uppi}。

另外,模板还为 `math-font` 提供了传统的 Type 1 字体 \opt{newtx}。
该选项会调用 \pkg{newtxmath} 宏包。
但是,如果西文字体已经使用了 OpenType 的 Times New Roman,
混用 Type 1 字体可能会导致问题,尤其是使用 \pkg{siunitx} 宏包时。
该选项还处于测试阶段,需要谨慎使用。

\subsection{参考文献}

参考文献支持使用 \BibTeX{} 或 BibLaTeX 生成。默认使用\BibTeX{}生成,用户可以根据需求在\file{hfutsetup.tex}中选择合适的方式。


\subsubsection{\BibTeX}

\BibTeX{} 是 \LaTeX{} 处理参考文献的传统的方式。
在本模板中需要载入 \pkg{natbib} 宏包,
然后按照“顺序编码制”格式选择参考文献格式风格:
\begin{latex}
  % HFUT学位论文样例 参考文献的著录格式
  \usepackage[sort]{natbib}
  \bibliographystyle{hfutthesis-numerical}
\end{latex}

\begin{latex}
  % gbt7714-2015 参考文献的著录格式
  \usepackage[sort]{natbib}
  \bibliographystyle{gbt7714-numerical}
\end{latex}

\begin{latex}
  % 备用方案
  \usepackage[sort]{natbib}
  \bibliographystyle{hfutthesis-graduate}
\end{latex}

并且在生成参考文献表的位置使用 \cs{bibliography} 设置 \file{.bib} 数据库的路径。
然后使用 \file{bibtex} 编译生成 \file{.bbl} 文件。
这里使用参考文献样式(\file{.bst} 文件)是在
\href{https://ctan.org/pkg/gbt7714}{\pkg{gbt7714}}
的基础上进行了少量修改。

顺序编码制的文内引用标注分为两种样式:
\begin{enumerate}
  \item 上标样式(\opt{super}), 如“张教授\textsuperscript{[2-3]}认为……
    形成了多种物理模型\textsuperscript{[7,9,11-13]}”。
  \item 与正文平排(\opt{inline}),文献序号作为叙述文字的一部分,
    如“张教授\textsuperscript{[4]}认为……根据文献[4]\textsuperscript{101-105}”。
\end{enumerate}
\textbf{模板默认为上标样式},也可以根据需要修改:

\DescribeOption{cite-style}
参考文献的样式。可选:\opt{super}(默认),\opt{inline}。
\begin{itemize}
  \item \opt{super}:上标样式
  \item \opt{inline}:与正文平排
\end{itemize}
\begin{latex}
  \documentclass[cite-style=inline]{hfutthesis}
\end{latex}
或者
\begin{latex}
  \hfutsetup{
    cite-style = inline,
  }
\end{latex}

\paragraph{与 \pkg{chapterbib} 的兼容性}
如果在每一章添加独立的参考文献,可以使用 \pkg{chapterbib} 宏包。
需要注意以下几点:
\begin{enumerate}
  \item \pkg{chapterbib} 宏包与 \pkg{natbib} 的载入顺序不重要。
  \item 根据 \pkg{natbib} 的文档,\opt{sectionbib} 参数(将参考文献的标题改为节标题)
    需要传给 \pkg{natbib} 而不是 \pkg{chapterbib}。
    模板作了设置,如载入 \pkg{chapterbib},\pkg{natbib} 就会有 \opt{sectionbib} 的效果。
  \item 每章的内容需要写在单独的 \file{tex} 文件,并且在主文档中使用 \cs{include},
    不带 \file{.tex} 后缀。
  \item 每章都要使用 \cs{bibliographystyle} 和 \cs{bibliography} 命令。
\end{enumerate}

用户可在\file{hfutthesis-numeric.bst}中的load.config函数中来自定义更详细的样式,表\ref{canshu}列出了一些常用的参数值,\# 1表示被启用,\# 0则不启用。
\textbf{默认值遵循《\href{http://xwgl.hfut.edu.cn/2021/0419/c1975a253949/page.htm}{合肥工业大学研究生学位论文写作规范}》,与国标的配置有部分出入。}
有关国标的配置可参考\href{https://ctan.org/pkg/gbt7714}{\pkg{gbt7714}}。

\begin{table}
  \caption{参考文献表样式的配置参数}\label{canshu}
  \begin{tabular}{lcl}
    \hline 参数值 & 默认值 & 功能 \\
    \hline uppercase.name & \#0 & 将著者姓名转为大写 \\
     sentence.case.title & \#1 & 将西文的题名转为 sentence case \\
    link.title &\#0 & 在题名上添加 url 的超链接 \\
    show.mark & \#0 & 显示文献类型标识 \\
    italic.jounal & \#1 & 西文期刊名使用斜体 \\
    show.missing.address.publisher & \#0 & 出版项缺失时显示“出版者不详” \\
    show.url & \#1 & 显示 url \\
    show.doi & \#1 & 显示 doi \\
    show.note & \#0 & 显示 note 域的信息 \\
    \hline
    \end{tabular}
\end{table}

\subsubsection{BibLaTeX}

BibLaTeX 是较新的参考文献处理方案。
需要在载入宏包时通过选择样式(\opt{style}),
并且在导言区使用 \cs{addbibresource} 设置数据库的路径,
最后在参考文献表的位置使用 \cs{printbibliography} 输出。

对于顺序编码制,文献引用标注默认为上标样式。
如果要切换为与正文平排,一种方法是使用 \cs{parencite},
另一种是载入 \pkg{biblatex} 时统一设置。
\begin{latex}
  \usepackage[bibstyle=hfutthesis-numeric,
  citestyle=hfutthesis-inline]{biblatex}
\end{latex}

使用 BibLaTeX 必须用 \file{biber} 编译生成参考文献表。
这里的样式是在 \href{https://ctan.org/pkg/biblatex-gb7714-2015}{biblatex-gb7714-2015}
的基础上进行了少量修改。
一些额外用法可以参考该宏包的文档。

\end{document}
